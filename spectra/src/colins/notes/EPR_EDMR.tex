\documentclass[svgnames]{article}    

\usepackage{graphicx}                 

% -- math                               
\usepackage{amssymb}
\usepackage{amsmath}
\usepackage{esint}
\usepackage{geometry}

% -- noindent
\setlength\parindent{0pt}

% -- images
%\graphicspath{{ }}                   

% -- tikz
\usepackage{pgfplots}
\pgfplotsset{compat=1.15}
\usepackage{comment}
\usetikzlibrary{arrows}
\usepackage[most]{tcolorbox}

\newtcolorbox{mybox}{
    enhanced,
    boxrule=0pt,frame hidden,
    colback=green!10!white,
    sharp corners
}

% -- figures


% -- figures
\usepackage{float}
\usepackage{caption}
\usepackage{lipsum}

% -- start 
\title{Electron Paramagnetic Resonance and Electrically Detected Magnetic
Resonance}
\author{Deval Deliwala}
%\date{}          

\begin{document}
\maketitle
\newpage 

\section{Electron Paramagnetic Resonance (EPR)} 

\subsection{Zeeman Effect} 

The intrinsic $|\frac{1}{2}, m_s \in \{\frac{1}{2}, -\frac{1}{2}\} \rangle$ spin of an electron gives rise to an
electron's magnetic moment. 

\[
    \vec{\mu}_e = \gamma \vec{S} = g_e \mu_B \vec{S}. 
\] \vspace{3px}

where $\mu_e$ is the magnetic moment of the electron, $\gamma$ is the
gyromagnetic ratio, $g_e$ is the Lande factor, and $\mu_B$ is the Bohr
magneton.\\ 

For an unperturbed electron, the Hamiltonian is given by 

\[
    H = -\vec{\mu} \cdot \vec{B} = -\gamma \vec{B} \cdot S.
\] \vspace{3px}

Therefore, the energy levels can be computed via 

\[
    E = -\vec{\mu}_e \cdot \vec{B} = g\mu_B \vec{S} \cdot \vec{B}.
\] \vspace{3px}

In a reference frame aligned with the magnetic field $\vec{B}$, the equation
simplifies to 

\[ E = -\vec{\mu}_e \cdot \vec{B} = g_e \mu_B S_Z B_0 \] \vspace{3px}

where $S_Z$ is the magnitude of the electron spin projected onto the $z$ axis
($+1/2$ or $-1/2$ ) and $B_0$ is the strength of the $\vec{B}$ field in the $z$
direction. Hence, there exists an energy difference between electrons of
different spin orientations. 

\[
\Delta E = g_e \mu_B B_0.
\] \vspace{3px}

This is known as the \textit{Zeeman Effect}. The energy splitting is
proportional to the applied magnetic field. Electrons in one of these states can
absorb and release energy to switch, or `flip' between spin states. \\

In EPR, this is achieved by exposing the electron to electromagnetic radiation.
The resonance energy required to flip an isolated electron is given by 

 \[
\Delta E = h\nu = g_e \mu_B B_0.
\] \vspace{3px}

However, this is for a free electron isolated from its surroundings. Unpaired
electrons in real materials deviate from this idealized situation. 

\subsection{EPR Response} 

In conventional EPR, a sample with paramagnetic centers is placed into a
microwave cavity of high quality/Q factor. A standing wave is set up in this
cavity to provide an oscillating $B$ field of frequency $\nu$. The cavity is
placed in a large slowly varying magnetic field. Resonance is detected when the
applied microwave radiation is absorbed by the sample, causing the Q factor of
the cavity to change. \\

The change in Q causes a change in the reflected power back to a diode in the
microwave bridge. An absorption spectrum is measured when the reflected
microwave power is plotted against the slowly varying magnetic field. \\

In a perfectly balanced system (infinite spin temperature) where there are an
equal number of spin up and spin down state, stimulated emission and absorption
would occur at the same rate and cancel each other. Hence, measuring the EPR
response is related to the spin polarization.  \\

The ratio of the number of spins in the up and down state for free and isolated
electrons can be approximated by 

\[
    \frac{N_+}{N_-} = \text{exp} \left( \frac{-g_e \mu_B B_0}{k_B T} \right)
    \approx 1 - \frac{g_e \mu_B B_0}{k_B T}. 
\] \vspace{3px}

Hence, increasing $B_0$ and decreasing $T$ increases spin polarization. This
ratio is approximately 1 if the Lande factor for the detect in question is
close to free electron, implying there is an equal balance of spin up and spin
down. This level of polarization is still high enough to detect approximately
$10^10$ paramagnetic sites. 

\subsection{Spin Relaxation}

As spins flip due to exposure to radiation satisfying the resonance condition,
they also interact with their local environment. Electron spins can return to
their original state via spin relaxation. \\

The time it takes for a spin system out of equilibrium to relax back to its
equilibrium is described by spin-lattice-relaxation time, $T_1$. Increased
electron coupling to the lattice results in lower $T_1$. Electron spins can
interact with each other causing them to precess at slightly
different rates, becoming de-phased over a spin-spin relaxation time, $T_2$. 


\subsection{Spin Orbit Coupling} 

The electron's magnetic moment is described by both terms, 

\[
    \vec{\mu}_e = g\mu_B \vec{S} + \mu_B \vec{L}.
\] \vspace{3px}


The electron `orbiting' the nucleus acts as a small current loop which in turn
generates its own magnetic field around the loop. However, in the electron
frame, the nucleus is the one revolving around \textit{it}, which alters this
magnetic moment. This is called \textit{spin-orbit coupling}. \\ 

The amount of spin-orbit coupling experienced is measured as a change in the
$g$. The $\vec{g}$ factor is a second-rank tensor that is anisotropic. 

\[
    \vec{g} = \begin{pmatrix}
        g_{xx} & g_{xy} & g_{xz} \\ 
        g_{yx} & g_{yy} & g_{yz} \\ 
        g_{zx} & g_{zy} & g_{zz} \\ 
    \end{pmatrix}.
\] \vspace{3px}

\subsection{Hyperfine Interactions}

Atomic nuclei can also possess spin angular momentum, though not all do. The
spin angular momentum, $\vec{I}$ alters the resonance condition of an electron
in a paramagnetic site. Unlike electrons, which all have spin $1/2$, the nuclear
spin depends on the \# of protons and neutrons in the nucleus. The magnetic
moment of the nuclear spin angular momentum is described by 

\[
    \vec{\mu}_N = g_N \mu_N \vec{I}
\] \vspace{3px}


where $g_N$ is the nuclear $g$ factor, and $\mu_N$ is the nuclear Bohr magneton.
A paramagnetic electron close to a nucleus with a nuclear magnetic moment
experiences a local magnetic field from the nucleus that differs from the
externally applied field. The interaction between electrons and nearby magnetic
nuclei is known as the \textit{nuclear hyperfine interaction} and is
characterized by a splitting of the measured EPR spectrum into two or more
lines. The Hamiltonian then becomes 

\[
    \hat{H} = \mu_B \vec{B} \cdot \vec{g} \cdot \vec{S} + \sum_{i}^{} \vec{I}_i
    \cdot \vec{A}_i \cdot \vec{S}. 
\] \vspace{3px}

where $\vec{I}_i$ is the nuclear spin operator for each nucleus and $\vec{A}_i$
is the hyperfine tensor for each nucleus. 

\subsection{EDMR} 

The most common EDMR experiments involve changes in spin dependent currents due
to recombination or tunneling. This has higher sensitive, with orders-of-magnitude 
less defects. \\ 

The detection scheme for EDMR relies on changes in the electrical
characteristics of a sample as it undergoes magnetic resonance. An electrical
connection must be made to the sample in the microwave cavity and a
current-to-voltage converter is used with a biasing circuit. \\ 

Measured EDMR spectra are generated by plotting the change in the electrical
current through the sample as a function of the slowly varied magnetic field,
$B_0$. The two primary spin dependent currents measured are spin-dependent
recombination (SDR) and spin-dependent trap assisted tunneling (SDTAT). 

\subsection{Spin Dependent Recombination} 

Electrons and holes can be captured by deep level defects within the bandgap.
When an electron in the conduction band and an electron in a paramagnetic defect
are polarized in the same direction, the conduction band electron cannot be
captured due to the Pauli-Exclusion Principle. \\

The electron in the trap can be flipped during magnetic resonance, converting
what was a triplet state into a
singlet state, making the previously forbidden transition possible. The
conduction electron can now be captured by the deep level defect and recombine
with a hole from the valence band. An increased number of recombination events
during magnetic resonance causes a measurable change in the steady state
current. \\ 

\subsection{Spin Dependent Trap Assisted Tunneling} 

SDTAT is an EDMR measurement that relies on the spin-dependent nature of VRH
currents, where electrons can hope between localized states via quantum
tunneling. Electrons cannot tunnel to a site with a same polarized electron
however. The electron can be flipped at magnetic resonance, thus allowing it to
tunnel to the next site. This change in tunneling current through the material
is measured during SDTAT and gives rise to the EDMR spectra associated with
defects. 

\newpage 
\section{NZFMR Mechanics} 

The density operator for a simple two-site model with an ensemble of $N$ spins
can be written as: 

\[ \rho = \frac{1}{N}\sum_{n=1}^{N} | \psi_N \rangle \langle \psi_N |.
\]\vspace{3px} 


Choosing an orthonormal basis set $\varphi_i$ allows us to define the density
matrix as: 

\[
    \rho_{i,j} = \langle \varphi_i | \rho | \varphi_j \rangle. 
\] \vspace{3px}

The system evolves coherently over time as the spins precess according to the
time-dependent Schrodinger equation and is described by the quantum Liouville
equation: 

\[
    \frac{\partial \rho}{\partial t} = -\frac{i}{\hbar} \left[\hat{H},
    \rho\right]. 
\] \vspace{3px}


For a system in which spin pairs are formed and removed over time, it is
necessary to include these phenomena in the equation governing the time
evolution of the system. This is achieved by using the stochastic Liouville
equation (SLE): 

\[
    \frac{\partial \rho}{\partial t} = -\frac{i}{\hbar} \left[H, \rho\right] -
    \frac{1}{2} \{\Lambda,\rho\} + \Gamma.
\] \vspace{3px}

where $\{\Lambda, \rho\}$ is the anticommutator of the projection operator
$\Lambda$ and $\rho$. The first term is the coherent time evolution with no
perturbations. The second term is a dissipative term that removes spin pairs
from the system. The last term is a generation term. 

\subsection{Electron-Hole SDR for NZFMR} 

An electron in the conduction band approaches a deep level defect with an
unoccupied electron near a nucleus of nuclear spin and an associated hyperfine
field.The free electron can be temporarily captured in an intermediate state
just below the conduction band where it can \textit{attempt} to enter the
defect. If the free electron and defect electron are a singlet state, the free
electron can be captured. Unlike in EDMR, the transition from triplet to singlet
does not come from impinging electromagnetic radiation, but from spin mixing of
the defect electron due to local magnetic fields. \\ 

If a singlet is formed due to spin mixing, the free electron drops into the deep
levels and recombines with a hole in the valence band, removing the electron and
hole from the system. This essentially annihilates the spins, hence an
additional requirement for defect and free electron spins to be antiparallel --
spin angular momentum conservation. 
















\end{document}
